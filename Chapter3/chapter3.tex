\chapter{Scalable User-centric cloud networking}
\ifpdf
    \graphicspath{{Chapter3/Chapter3Figs/PNG/}{Chapter3/Chapter3Figs/PDF/}{Chapter3/Chapter3Figs/}}
\else
    \graphicspath{{Chapter3/Chapter3Figs/EPS/}{Chapter3/Chapter3Figs/}}
\fi

In this chapter we explore application of the SDN technology in personal cloud
computing applications. In the recent years, the development in cloud computing
applications provided a number of popular Internet-wide services that allow
users to interconnect at the application layer and share information. Through
this approach users bypass the restrictions imposed in the Internet by deployed
middleboxes, at the cost though of exposing their information to third party
cloud service providers and reduced performance and efficiency. 

Using SDN technologies, we design an architecture that allows users to
interconnect their devices with minimum interactions with the cloud
infrastructure.The propose architecture deploy an openflow-enabled bridge and a
local controller on each device. Each controller by default forwards packets as
normal on the local network. The forwarding logic, though,  is augmenting
through a distributed coordination protocol which permits nodes to negotiate
possible connection opportunities and establish ad-hoc tunnels, enabling as a
result an Internet-wide distributed control mechanism. At the core of
our design, we tranform the naming service host abstraction. Each device
acquires a global domain name, while each name resolution triggers a connection
engine, that tries to find the best possible bidirectional channel between the
two nodes. The naming service uses the established
DNSSEC extension, providing a fully authenticated and secure control
mechanism among \signpost and applications. Further, the distributed nature
of the naming hierarchy in the Internet permits seamless control distribution.

In order to understand the impact of the proposed architecture we develop a
strawman implementation, named {\it Signpost}. Signpost implements the core of
the control logic of the proposed architecture. Additionally, it integrades a
number of network tactics of established tunneling and notification mechanisms.
Currently \signpost provides integration of the main architecture with SSH, OpenVPN,
TOR, Privoxy, Multicast-DNS and Nat punching. 

In the chapter we present in Section~\ref{sec:signpost-introduction} the
motivation for this work, followed then by the key observations for our design
in section~\ref{sec:signpost-design}. Following the results of our
observations, we present in section~\ref{sec:signpost-architecture} the
architecture of our proposed proposed strawman implementation of the controller
and the details of the tactics. Finally, in Section~\ref{sec:signpost-evaluation}
we present a number of micro-benchmark tests for our system and conclude in
Section\ref{sec:signpost-conclusion}.

\section{Introduction}\label{sec:signpost-introduction}

Present the gap in network understanding between the clients and the ISP and
present a tool that allows to bridge it.

\section{Enabling edge user-driven connectivity}\label{sec:signpost-design}

\section{Signpost Architecture}\label{sec:signpost-architecture}

\section{Evaluation}\label{sec:signpost-evaluation}

\section{Conclusions}\label{sec:signpost-conclusion}
%\section{Second Section}
%\markboth{\MakeUppercase{\thechapter. My Second Chapter }}
%and here I write more ...
%
%\subsection{first subsection in the Second Section}
%... and some more ...
%
%\subsection{second subsection in the Second Section}
%... and some more ...
%
%\subsection{third subsection in the Second Section}
%... and some more ...

% ------------------------------------------------------------------------

%%% Local Variables: 
%%% mode: latex
%%% TeX-master: "../thesis"
%%% End: 
