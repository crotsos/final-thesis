\chapter{Scalable User-centric cloud networking}
\ifpdf
    \graphicspath{{Chapter3/Chapter3Figs/PNG/}{Chapter3/Chapter3Figs/PDF/}{Chapter3/Chapter3Figs/}}
\else
    \graphicspath{{Chapter3/Chapter3Figs/EPS/}{Chapter3/Chapter3Figs/}}
\fi

In this chapter we explore application of the SDN technology in Cloud Computing.
The availability of cheap cloud computing resources, boosted the development of
a wide ecosystem of applications that aim to fulfil the needs of users to handle
efficiently and ubiquitly personal information.  We term this use case of cloud
computing as Personal Cloud Computing (PCC). Applications like facebook,
dropbox, google services~et.al.  abstract online user identity among devices,
associate it with information, which is disseminated based on the user policy.
Such services manage to simplify the way we use computers, but most of the times
they employ mechanisms that undermine the security and privacy of users, while
use network resources inefficiently. 

We propose an Internet-wide overlay network architecture which enables PCC, while
addressing the aforthmentioned shortcomings. The architecture builds around an
\of controller running on end-hosts, which is enhanced with the ability to
control several off-the-self tunneling software. The proposed architecture employs
a distributed network connectivity negotiation protocol between end-hosts, that
scales the complexity of end-system orchestration through control ditribution.

From the user perspcective, the system achieves backward compatibility with
existing applications through the integration of the  architecture on the
network layer of the network stack. As a result, legacy aplications can take
advantage of the architecture by communication over a specific local
network subnet. Additionally, the proposed system exposes a simple API to
applications over the naming system, which we call {\it ``Effectful Naming''}.
Devices and user identities are represented as domain names , and an end-to-end 
path is establish through a simple name lookup. This mechanism is secured by the 
\dnssec namint service extensions. 

% The propose architecture utilizes \of-enabled bridge and a
% local controller on each device. Each controller by default forwards packets as
% normal on the local network. The forwarding logic, though,  is augmenting
% through a distributed coordination protocol which permits nodes to negotiate
% possible connection opportunities and establish ad-hoc tunnels, enabling as a
% result an Internet-wide distributed control mechanism. 
% At the core of
% our design, we  the naming service host abstraction. Each device
% acquires a global domain name, while each name resolution triggers a connection
% engine, that tries to find the best possible bidirectional channel between the
% two nodes. The naming service uses the established
% DNSSEC extension, providing a fully authenticated and secure control
% mechanism among \signpost and applications. Further, the distributed nature
% of the naming hierarchy in the Internet permits seamless control distribution.

In order to understand the feasibility and performance of the proposed
architecture we develop a strawman implementation, named {\it \signpost}.
\signpost implements the core of the control logic of the proposed architecture.
Additionally, it integrates in the architecture various network connection and
notification services.  Currently, \signpost supports SSH, OpenVPN, TOR,
NAT-punch and Privoxy connection mechanisms, while it can propagate
Multicast-DNS notifications across devices.

In the chapter we present in Section~\ref{sec:signpost-introduction} the
motivation for this work, followed then by the key observations for our design
in section~\ref{sec:signpost-design}. In
Section~\ref{sec:signpost-architecture}, we present the
architecture of our strawman implementation and it integration with existing
software. Finally, in Section~\ref{sec:signpost-evaluation}
we present a number of micro-benchmark tests for our system and conclude in
Section\ref{sec:signpost-conclusion}.

\section{Personal Clouds}\label{sec:signpost-introduction}

In the recent years, the increase in the number of networked devices per user
has created a significant information and resource management problem for the
user.  Each device usually represents a different aspect of the user interaction
with technology, but the distincts roles of each device can mix, e.g.
smartphones are primarily a communication device but they are used as
entertainment device too and users are interested in mechanisms that would allow
them to share data with the home NAS.  Ultimately, users want
to manage the ensemble of devices as a single control entity and virtualize
resources. This requirement motivated the development of a large number of
applications that enable seemless inter-device cooperation. Protocol designers
have developed protocols that allow functionalities like remote desktop, file
sharing, remote login, remote printing etc. over the network and across devices.
Additionaly, in order to reduce the configuration burden, the network community
has also proposed a number of automation mechanisms.  Multicast-DNS, Universal
Plug and Play and ZeroConf allow a user upon network connection to review most
available shared services and connect to them with a single click. We term {\it
  ``Personal Cloud``}, the resource sharing and control abstraction that the user 
can exercise among its devices. Personal cloud is an efficient way for users
to
embrace technology and take full advantage of it. Unfortunately, interconnecting 
devices across the Internet has become increasingly difficult.
% , mainly due to the
% way that the Internet evolves.

\subsection{Disconnected Internet}

Internet is an excelent example of a dynamic system that manages to adapt to the
evolving demands of its users and modify its functionality and design in order
to accomodate evolvability.  In the early days of the system, its designing
boards set as a fundamental goal to promote simplicity and openness, in order to
expand the user base worldwide. As a result, Internet protocols achieved pretty
quick extensive systems support, while hosts could use Internet services simply
by connecting to a network with an Internet gateway. During that period, the
Internet remained a large wide area network interconnecting research institutes
and its main use cases focused on providing various types of asynchronous
communication. Through the years, the evolution of Internet use cases
highlighted a number of limitation on the design of the system, a number of
which related to security and performance.

In order to address these limitations in a backward compatible manner a number
of design modifications were introduced in the Internet design. One of the most
popular technique is the deployment of middleboxes.
Middleboxes violate in a number of ways design principles of the Internet, but
they provide an effective framework to insert in the network
new functionality that can address some of its limitation. 
NATboxes permit Internet to overcome the shortage of IP
addresses, WAN optimizers optimize utilization of 
udnerprovisioned links and firewalls secure critical networks against
malicious hosts. Unfortunately, because this approach violates a number of
design patterns of the Internet, like the end-to-end principle, middlebox usage has
redefined the functionality of the Internet. A number of papers have quantifies
their impact. In~\cite{blah} authors blame middlebox
functionality for the inability to introduce new transport protocols, while
in~\cite{blah} authors presents some details on how middleboxes handle corner cases
of protocol implementation. \todo{ describe some efforts to overcome
  middleboxes.}

An important impact of middlebox deployment and development is the reduced
connectivity introduced in the Internet. Home networks host a number of
hidden devices which are inaccesible from the Internet, while enterprise
networks deploy strict security policies that permit their users to access only
specific network applications. Personal Cloud deployment across the Internet is
heavily restricted by the luck of openness.

% §develop and deploy various protocol modifications that tried to bypass these
% design flaws.  Unfortunately, such systems tend to introduce stricter
% assumptions on how the network should function, and reduce to a great extend the
% openess of the Internet.There are primarily two major classes of engineering
% modifications that reduce network connectivity: performance enhancing
% middleboxes and edge-network security policy reinforcement.  
% Performance
% enhancing middleboxes are network forwarding devices, installed in the network,
% that aggregate information from multiple layers of the TCP/IP stack and modify
% packet content or forwarding logic. 
% This restriction in openess of some parts of
% the Internet is a vital reason for the inability of the network community to
% establish Private Clouds across the Internet.  

\todo{Internet is disconnected due to policies as well as middleboxes. This
  reduces the ability of users to interconnect devices. Although the applications
exist, we cannot use them easily across the network.}

\subsection{Reconnecting Internet} 

\todo{The Internet remains disconnected due to policies on the edges. The
  core of the Internet enforces very little policying. As a result the problem
  of interconnecting devices across ther Internet is not a limitation of the
  network, but a luck of mechanism to achieve this. }

Although Internet reachability is reduced, the system remains highly performant
and connected; it interconnect billion of users and transports terrabytes of
information everyday. This is achieved due to the predominant mode of
operation of Internet services. Hosts in the common case connect to
well exposed services, that provide access to popular information. 
This model, which is called Client-Server model, is a natural occurance of the
power law behaviour of various social and engineering mechanisms. Internet
network graph is a power law graph, due to its hierarchical structure that
allows scalability, while information popularity exhibits strong power law
characteristics. 

Personal cloud computing is not a good match for the
Client-Server model. Personal devices most commonly are connected to networks
that are optimised for outgoing activity, while the Internet is not able to
acomodate well-connected services for every host. More
specifically, NATed networks that can scale connectivity on the edges of the
network require manual configuration by the administrator in order to permit
hosts to expose services, while a number of edge networks, like mobile and
enterprise networks, restrict publicly accessible services on connected
hosts. In order to overcome these restrictions a number of approach has been
proposed by the research community as well as the industry. We group these
solutions in the following two categories:

\paragraph*{User-managed decentralised Personal Cloud}: A number of 
frameworks have been proposed over the years that use publicly available
services and provide biderectional connectivity between devices. We are
considering in this category tunneling software, like OpenVpn and SSH, and 
NAT and firewall puncing mechanism, like STUN. Although such mechanisms are
effective in a number scenarios, a number of assumption and configuration
requirements dim them inappropriate for inexperinced users. As we have already
discussed in Section \ref{blah} average Internet user is willing to engage in
configuring netowrk services as long as they can be defined as short simple
management tasks. Establishing connectivity using existing user managed mechanisms is not
staightforward. Functionality contains assumptions on the connectivity of the
environment and users have to resolve to ``try and error'' approaches to check
which mechanism can be effective in a specific environment, while users are
required to engage in a number of prior configurations in a number of different
subsystems of the network. 

In order to exemplify we can engage in a detailed description of the steps for a
user to establish connectivity using the ssh service. Before the user is able to
connect to a computer using SSH, he has to configure the service on his local
mahine, configure security credentials on the server and configure any firewall
and nat boxes. Upon connection, the user has to run the SSH client, choose the
ports he is interested to forward and configure the connecting software to use
the ports expose by the ssh client. This connectivity is subject to the ability
of the user in the remote network to use the SSH protocol on the preconfigure
port. 


\paragraph*{service-assisted Personal Cloud}:

% According to wikipedia~\cite{wiki-cloud}, the definition of {\it ``Cloud''} is:
% \begin{quote}
%   Cloud computing is the use of computing resources (hardware and software) that
%   are delivered as a service over a network (typically the Internet). 
% \end{quote}
% The {\it``Coud''} has been a buz world that receive increased attention within
% the recent years.  The concept is not novel for the computer science community,
% and can be placed under the generic engineering technique of abstraction and
% virtualisation. Its popularity though can be mapped to theeffectiveness of the
% abstraction as well as the increased interest in distributed large data
% processing infrustructure.  Interestingly, the initial idea for cloud computing
% aimed to create a small scale economical market that would take advantage of
% iddle data center resources, but the model became so successful that companies
% create new data center infrustructure solely to support this service. 

% Our work in this chapter aims to revisit the design of cloud applications that
% target end-users and provides them the ability to store and share
% information. Such application usually consist of 2 layers of abstraction, which
% oftenly are not controlled by the same entity. In the lower layer of the system
% we find the infrastructure abstraction layer, which is usally termed as
% {\it Infrustructure as a Service}. This service is controlled by the
% entity that controls the data center and ensures and enforce the required resource
% allocations over the available hardware. Usually, this relies on the management
% of a large number of commodity servers running an OS virtualisation framework,
% like Xen, and configure the network in order to match the resource
% virtualization mechanisms. At the upper layer of the system, we find the service
% abstraction layer, which is also called as the {\it Software as a Service} or
% {\it Platform as a Service}. This layer provides the resulting abstraction to
% the user, and it main focus is to translate the required functionality
% by the user into a number of distributed computations that can be distributed
% over the abstraction of machines that the lower layer provides. In most of the
% cases, the end-user of a cloud application is usually exposed  directly only to
% the upper layer of the afortmentioned abstractions. 

An alternative approach which has been highly succesful in the recent years
employs the Client-server model to establish connectivity between devices, using
a third party service. Such systems have been highly effective, while the high
high popularity of cloud services has reduced significantly the cost, providing
significant resources at low costs and good performance. 
% Cloud computing mechanisms have managed to provide a sufficient platform that
% can accomodate the processign requirements for a large number of applications,
% while it also provides ways to scale systems very fast in order to support
% emerging applications. The main architectural approach that has been used in
% order to achieve is to direct users to a centrally controlled system. User copy
% their data to the public cloud infrustructure and interact with the public API
% of the service. The service provider can post analyse the user data in order to
% create the intermediate data required in order to provide timely responsiveness
% for its service. 
Although this approach is deemed succesful, especially given
the impact such applications have among users, its properties can be
characterised ambivalent. A number of properties that cloud services provide
may have a significant set of user implications. 

\begin{itemize}
  \item{\it authentication}: A number of cloud applications provide an effective
       framework to control information dissemination between devices as well as
       users.  Users are required solely to verify that an online identity is a
       valid destination for an information quantum and the service ensures
       secure dissemination of information based on the user policy.  This
       mechanism though introduced privacy concerns.
       In~\cite{Krishnamurthy2009} authors reports cases of online OSN that leak
       information to ad services which allows them to detect and characterize
       individuals, while facebook has openly verified the existence of such
       services~\footnote{\url{http://en.wikipedia.org/wiki/Facebook_Beacon}}. 

% \paragraph*{authentication}: A number of cloud applications provide an effective framework
% to control information dissemination between devices as well as users. 
% Users are required solely to verify that an online identity is a valid destination for
% an information quantum. The platform will ensure that the information will be
% accessible only to users that have the valid credentials for that account in any
% time, easing the information access control from end-users. This mechanism has
% been further used by other application to offload their authentication mechanism
% to a third party cloud service. This mechanism though has been reported to be
% abused by some service providers. In~\cite{Krishnamurthy2009} authors reports
% cases of online OSN that leak information to ad services which allows them to
% detect and characterize individuals. This was recognized by facebook as a paid
% service provide to ad services and removed after wide privcacy concerns by
% users~\footnote{\url{http://en.wikipedia.org/wiki/Facebook_Beacon}}. The wide
% adoption of such authentication mechanisms raises concerns for the control of
% the privacy of users.  
% On the other hand though, it has been reported that some cloud
% applications have engaged in leaking information on users to third party
% application. A study in~\todo{add reference} reports that the social network
% facebook leaks identity hints to advrtising platforms in order to enhance their
% ability to characterise users, a functionality which is privacy envasing for
% some users. 

\item {\it performance}: Cloud services manage to deliver user
acceptable performance to million of users across the world. Google claims that any user is in
range of a few thousand of mile of at least a signle google datacenter. Although
this approach provides a user satisfying service performance, it is wasteful in
processing resources; Cloud services under utilize rich edge resources. Two
devices that are behind the same subnet and communicate using a cloud service 
have will suffer high RTT latency, while public cloud storage cost and
performance is orders of magnitude higher that local network NAS. 
Finally, a number of measurement studies have highlighted a significant impact
on the 95th quantile network performance of cloud services, due to latency and packet
losses incurred by the Internet~\cite{Wittie2010}. 
% \paragraph*{performance}: A number of cloud services manage to deliver user
% acceptable performance to million of users across the world. This is achieved to
% a great extend by careful design of data processing pipelines, that paralelize
% processing logic and preemptively calculate results for future user requests.
% Additionally, large service providers develop global distribution networks with
% multiple entry points close to the user.  Google claims that any user is in
% range of a few thousand of mile of at least a signle google datacenter. Although
% this approach provides a user satisfying service performance, it is wasteful in
% processing resources; Cloud services under utilize rich edge resources. Two
% devices that are behind the same subnet and communicate using a cloud service 
% have to communicate through the datacenter infrustructure of the provider, 
% increasing significantly latency. Additionally, public cloud storage cost is two
% orders of magnitude more expensive that the cost to maintain data on your local
% disk. 
% Finally, a number of measurement studies have highlighted a significant impact
% on the 95th quantile performance of cloud services, due to latency and packet
% losses incurred by the Internet~\cite{Wittie2010}. 

\item {\it cost}: Free cloud services have an interesting 3-party economical
      model that uses advertisment as a mechanism to sustain the cost of the
      infrustructure. As a result, users are able to enjoy high quality services
      with minimum costs. The providers though, because of the free nature of
      the service are not wiling to provide any SLA's to users. If a part of the
      service is compromised and some information is leaked, then the service
      provider bears no obligation towards affected users. Such costs are not
      directly observed by the end-user, but they may impact significantly his
      everyday life. 

% \paragraph*{cost}: Free cloud services have an interesting 3-party economical
% model that uses advertisment as a mechanism to circulate capital and sustain the
% cost of the infrustructure. As a result, users are able to enjoy high quality
% services with minimum costs. The SLA's though, because of the free nature of the
% service are not wiling to provide any SLA's to users. If a part of the service
% is compromised and some information is leaked, then the service provider bears
% no obligation towards affected users. Such costs are not directly observed by
% the end-user, but they may impact significantly his everyday life. 

\item {\it Availability}: Cloud services run on well connected infrastructures
      with a large number of network engineers ensuring security and
      performance.  Any device with Internet connectivity is able to connect to
      the cloud service, without any special configuration of the OS or the
      Network policy. Cloud services function as an always accessible cache
      between devices and users.  Although, the centralised design of such
      services makes it impossible for devices to interconnect when Internet
      connectivity is restricted. Two users behind the same firewall will never
      be able to exchange a file over dropbox, if the network policy forbids any
      connection with the servers of the service. 
% \paragraph*{Availability}: As we have already discussed in the Introduction of
% this thesis, Internet has reduced significantly the bidirectional connection
% property of Internet hosts. The deployment of performance-enhancing middleboxes
% makes highly impossible for a device to expose an Internet-wide service.
% Additionally, the large number of malicious nodes that try to exploit security
% vulnerabilities, is counter intuitive for users. Cloud service providers address
% this problem in a highly effective manner. Their network default policy is
% permit, while a handful of network engineers are constantly monitoring the
% security of the datacenter and ensure high availability of the system and data
% integrity.  As a result, any device with Internet connectivity is able to
% connect to the service, without any special configuration of the OS or the
% Network policy. Cloud services function as a proxy between devices and users,
% that provides them connectivity in most cases. Although, the centralised
% design of such services makes it impossible for devices to interconnect when
% there isn't any Internet connectivity. Two users behind the same firewall will
% never be able to exchange a file over dropbox, if the network policy forbids
% any connection with the servers of the service. 
\end{itemize}

\todo{A large set of protocol to coordinate distributely home networks. }

\section{Enabling edge user-driven connectivity}\label{sec:signpost-design}

\todo{\dnssec description}

\todo{Effectful naming}
\todo{How \signpost looks like to the outside}

\section{Signpost Architecture}\label{sec:signpost-architecture}

\todo{Connection Engine}

\todo{Tactic life cycle}

We propose a novel overlay network architecture that enables PCC while
minimizing the interaction with the cloud infrustructure. 
The propose architecture utilizes \of-enabled bridge and a
local controller on each device. Each controller by default forwards packets as
normal on the local network. The forwarding logic, though,  is augmenting
through a distributed coordination protocol which permits nodes to negotiate
possible connection opportunities and establish ad-hoc tunnels, enabling as a
result an Internet-wide distributed control mechanism. At the core of
our design, we tranform the naming service host abstraction. Each device
acquires a global domain name, while each name resolution triggers a connection
engine, that tries to find the best possible bidirectional channel between the
two nodes. The naming service uses the established
DNSSEC extension, providing a fully authenticated and secure control
mechanism among \signpost and applications. Further, the distributed nature
of the naming hierarchy in the Internet permits seamless control distribution.

In order to understand the impact of the proposed architecture we develop a
strawman implementation, named {\it Signpost}. Signpost implements the core of
the control logic of the proposed architecture. Additionally, it integrades a
number of network tactics of established tunneling and notification mechanisms.
Currently \signpost provides integration of the main architecture with SSH, OpenVPN,
TOR, Privoxy, Multicast-DNS and Nat punching. 


\section{Evaluation}\label{sec:signpost-evaluation}

\section{Conclusions}\label{sec:signpost-conclusion}
%\section{Second Section}
%\markboth{\MakeUppercase{\thechapter. My Second Chapter }}
%and here I write more ...
%
%\subsection{first subsection in the Second Section}
%... and some more ...
%
%\subsection{second subsection in the Second Section}
%... and some more ...
%
%\subsection{third subsection in the Second Section}
%... and some more ...

% ------------------------------------------------------------------------

%%% Local Variables: 
%%% mode: latex
%%% TeX-master: "../thesis"
%%% End: 
