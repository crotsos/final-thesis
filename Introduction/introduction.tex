%%% Thesis Introduction --------------------------------------------------
\chapter{Introduction}
\ifpdf
    \graphicspath{{Introduction/IntroductionFigs/PNG/}{Introduction/IntroductionFigs/PDF/}{Introduction/IntroductionFigs/}}
\else
    \graphicspath{{Introduction/IntroductionFigs/EPS/}{Introduction/IntroductionFigs/}}
\fi

In this thesis I am planning to support the following position:

\begin{quotation}
In the recent years, the Internet has met an incredible development in terms of
infrastructure as well as connectivity. This development has been driven by two
main causes: the significant reduction of the cost of network-enabled device and
the subsequent widespread adoption from users and
the wider utilization from companies of the Internet as a medium to interconnect and
expand their market. Although significant, Internet development has been highly
asymetric. The capacity of the core of the Internet has increased
exponentially over the recent years and routing has become stable. 
As a result, the network bottleneck has moved to the edges of the Internet, a
point where link upgrades have a significantly higher aggregate cost and access 
technologies have seen little innovation.
In this work, I argue that edge networks performance is further reduce due to
a great extend to the way many internet scale protocols are deployed.
Further, I believe that in order to handle more efficiently
resources in the edges, functionality of network devices should be customized to
the needs of the specific environment and network processing should be developed
over newer abstractions, that match the flow primitives of the specific
environment. In this work I focus in the case of home networking and I present
a number of novel network designs that leverage the ability to control and use
home networks.
\end{quotation}

In the introduction I think it is important to mention the following points:
% I am planning to present 2 main points:
\begin{itemize}
\item The core of the Internet is highly optimized and can perform really
well the task of packet forwarding end-to-end. Although the programmability
of the core is restricted due to the low cost principle and the high
multiplexing of network connections. On the other hand, edge
networks exhibit lower connection multiplexing which make programmability 
to be handle using restricted resources. Further, in the edges we are able 
to integrate in the
packet processing process useful input from the users. There a number of
measurement studies that present this differentiation both for the ADSL/cable
(Netanalyzer, bufferbloat, ADSL measurement studies) world, as well as 3g 
(e.g. 3gtest).
\item The concept of network programmability is not a new concept. It has already
been discussed in different forms (e.g. ATM controllers/switchlets, active networks) 
which though never manage to get deployed in the real world. I need to discuss
for the most significant cases, why these mechanisms failed to meet their
requirements and why the SDN approach solves some of their problems. 
\item An additional problem that I should discuss in this chapter is the
different abstractions perceived by network applications and service
providers. This difference in abstractions result in a significant loss of
information that could potentially be used by both sides in order to optimize
network utilisation.

\end{itemize}
%%% ----------------------------------------------------------------------


%%% Local Variables: 
%%% mode: latex
%%% TeX-master: "../thesis"
%%% End: 
