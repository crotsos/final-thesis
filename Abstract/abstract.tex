
% Thesis Abstract -----------------------------------------------------


%\begin{abstractslong}    %uncommenting this line, gives a different abstract heading
\begin{abstracts}        %this creates the heading for the abstract page

What is my main moto? <<evolving control on the edges, can optimize the current
bottlenect in the network, nd thus improve overall performance, but also it can
scale sufficiently. >>

The evolution of human communication needs has been radical in the recent years.
Internet connectivity has become nowadays vital from many aspects of society, 
and Internet accesibility is slowly recognised as a fundamental human right. 
Network innovation hasn't been proportional. Network perfromance requirements
are enhanced, but modern networks have become highly complex, as well as 
network performance requirements. 
Although, link rates have inceased significantly, the complexity of modern 
networks hardenis the optimization task. A number of measurement analysis papers
have recently moved the netowkr bottleneck close to the edges of the network. 
This is a direct concequence of the low-cost requirement of computer networks. 
In order to enhance the edges, ISPs need to invest a large amount of money 
in order to replace the connection medium and upgrade equipment in the last 
mile. 

Keeping in accordance with the  end-to-end principle of computer networks, a 
approach would be to develop more efficient protocols. Unfortunately, 
the requirement for fast connectivity at low cost, has assimilate to the network
a number of 
strong assumtpions, that make it impossible to develop and propose new network
protocol that address aforthmentioned problem. An alternative approach to the 
problem is to provde evolution through the control plane. Such approaches have
been explored in the past without a lot of adaption. A recent development in the 
field is called {\it SDN} and gains a lot of interest from the comunity. 

In my thesis, I will firstly present a set of evaluation platform and results
that try to understand the impact of the SDN paradigm, and especially its popular
implementation {\it OpenFlow}. The result show that the protocol implementation are not yet sufficiently mature to be deployed across the network. Although, 
software implementations of the protocol are exceptionally efficiently. 

This observation drives my exploration on the possibilities of deploying 
OpenFlow in the edge of the network, close to end-users. 




\end{abstracts}
%\end{abstractlongs}


% ----------------------------------------------------------------------


%%% Local Variables: 
%%% mode: latex
%%% TeX-master: "../thesis"
%%% End: 
