
% Thesis Abstract -----------------------------------------------------


\begin{abstractslong}    %uncommenting this line, gives a different abstract heading
%\begin{abstracts}        %this creates the heading for the abstract page

Computer network technologies have become the catalyst for the development of
the current digital revolution. Nonetheless, their increased adoption has
highlighted novel challenges in their functionality. The increasing popularity
of the network abstraction has unveiled a series of scalability limitations in
the design of the predominant protocols. In addition, the increase in the number
of network applications has further complicated the definition of network
performance, i.e.~management flexibility, resource control, connectivity. 

The thesis of this dissertation contends that the perfomance scalability
limitation of data plane protocols can be alleviated by redesigning the control
plane architecture of networks. We argue for the development of speciailised
control designs tailored to the requirements and the opportunities of the
deployed environment, taking advantage of the uprecedented capabilities provides
by the SDN paradigm for programmable control. 

Towards this goal, this dissertation explores three specific cases of scaling
networks for improved perfromance by evolving network control. Firstly, we
present two generic tools that enable characterisation of the scalability and
performance of programmable network control. Using these tools, we characterise
the scalability of the control plane performance for a series of production
\of-enabled forwarding devices and the performance of hierarchical control
schemes in a datacenter environment. Secondly, we present a control plane design
that scales resource and access management for the home network environment.
Finally, we present a control plane design providing Internet-wide naming and
connectivity. 

%\end{abstracts} 
\end{abstractslong}



% ----------------------------------------------------------------------


%%% Local Variables: 
%%% mode: latex
%%% TeX-master: "../thesis"
%%% End: 
