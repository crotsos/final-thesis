\def\baselinestretch{1}
\chapter{Conclusions and Future Work} \label{sec:conclusions}
\ifpdf
    \graphicspath{{Conclusions/ConclusionsFigs/PNG/}{Conclusions/ConclusionsFigs/PDF/}{Conclusions/ConclusionsFigs/}}
\else
    \graphicspath{{Conclusions/ConclusionsFigs/EPS/}{Conclusions/ConclusionsFigs/}}
\fi

\def\baselinestretch{1.66}

This chapter concludes the dissertation by summarising the work it described,
and noting areas in which further work is required.

\section{Summary}

This dissertation has addressed issues of control plane performance scalability
using the Software Design Networking paradigm.
Chapter~\ref{s:introduction:introduction} began by motivating the significant
bottleneck of current network technologies on the control plane. Nontheless, we
argued that control plane performance has multiple aspects (e.g. responsiveness,
usability, security), which have variable importance and depend highly on the
deployment environment. In was concluded that an effective control plane
architecture must take under consideration such requirements, as well as, the
properties of the environment during the design of efficient control plane
architectures. 

Chapter~\ref{ch:background} then considered background and related work to the problem of network
control. We provided a bottom-up design discussion on available network control
mechanisms. We, initially, elaborated on the architecture of current network
devices and discussed the design of the integration between the control and data
plane in a single devices. Furthermore, we reviewed the current production-level
network control protocols and mechanisms for the data link and network layers
and it was argued that their ability for responsive and user friendly control is
reduced, especially as data-plane rates increase exponentially. Furthermore, we
presented a series of experimental approaches which combat a series of network
control limitation and provide flexible, distributed and evolvable control.
Namely we present Active Network, Devolved Control of ATM Network and Software
Defined Networking. In order to highlight the opportunities provided by these
mechanisms, we concluded this chapter with an extensive presentation of novel
control applications build on top of the SDN paradigm. 

The bulk of the contribution of this dissertation was reported in the following
three chapters. Chapter~\ref{sec:sdn_scalability} analysed the elementary
scalability of the SDN paradigm. In this chapter, we presented two measurement
platforms, enabling SDN experiments to evaluate the performance of SDN
architectures. More specifically, we presented \oflops, a high-precision
hardware-accelerated \of switch measurement platforms which enables
experimenters to understand the performance behaviour of \of-enabled devices,
and \sdnsim, a lightweight network high-precision simulation and emulation
framework. Using \oflops, we develop and run a series of tests characterising the
elementary functionalities of the \of protocol and detect significant difference
between switch protocol implementations which can affect significantly the 
performance of an \of-based control architecture. Motivated by the variability
between the implementations  of the \of protocol, we have developed the \sdnsim
platform. The \sdnsim platform provides a high precision experimentation
environment, which allows users to implement their control logic and traffic
models and evaluate the performance of the system using specific switch
performance profiles. Using \sdnsim, we replicate the functionality of a small-size 
datacenter networks and evaluate the effectiveness of distributed hierarchical
control-plane architectures, finding minimal impact in the data-plane
performance.  


% considered control timescale traffic engineering,
% .e. dealing with connections, concentrating on the TCP and RTP protocols. It
% began by demonstrating that current approaches to congestion control in TCP can
% fail in extreme cases to ensure that all users achieve reasonable goodput
% through the network. It also showed that even if such failure does not occur,
% TCP allocates resource in a highly variable and potentially unfair manner.
% To alleviate these problems, admission control and specifically implicit ad-
% mission control was proposed. The potential impact of this was discussed,
% followed by design considerations for such a system. Implementation of such a
% system in the Linux operating system was then presented, demonstrating the
% feasibility of this approach. Simulation work reporting an implementa- tion of
% implicit admission control based on measured traffic statistics in the NS
% simulator followed, and showed that implicit admission control for TCP
% substantially improves the performance of the network at times of overload.
% Finally, implementation of an RTP-ECN-proxy demonstrated the feasibility of an
% alternative to admission control. The presented mechanisms were shown to improve
% the performance of the network for users, and the controllability of traffic
% within the network for operators.
Chapter~\ref{sec:homework} elaborated on the problem of management scalability
in modern home networks. More specifically, using ethnographic and measurement
studies of the home setting, we identified a significant mismatch between the
user requirement and understanding of network functionality and the existing
technologies. Motivated by this observations, we redesign the home-router,
implementing a series of protocol modifications which enhance user control and
bridge the user network perception with the underlying functionality.
Additionally, we extended our proposed architecture and presented a distributed
resource control mechanism which integrates users application-level requirements
with the ISP policy, in an effort to develop a user-friendly control scheme for
the last-mile bottleneck in current residential broadband networks. We presented
the development of a strawman implementation of our system and verified that the
architectures incurs minimal impact on network functionality and the protocol
modifications remain backwards compatible with a number of popular devices,
OSes and applications.  

Chapter~\ref{sec:signpost} discussed Internet-scale connectivity scalability.
Through the work of this chapter, we aimed to develop a decentralised and
Internet-wide Personal Cloud between the devices of users. In this chapter we
described the limitations of existing Cloud platforms in terms of usability and
privacy and argued that an evolved control plane for end-hosts can address such
limitations and support all required functionality.  We proposed the \signpost
architecture, an evolved control plane for end-user devices providing secure,
continuous and decentralised inter-device connectivity. \signpost enables local
device controllers to establish a global control channel, test and negotiate
connectivity capabilities between devices and establish ad-hoc secure end-to-end
data plane paths between devices using off-the-self connection establishing
mechanisms. Furthermore, we presented an implementation effort of the \signpost
architecture, enabling support for \openvpn, TOR, SSH, Privoxy, NAT punching and
DNS-SD functionality. Using the \signpost strawman implementation, we ensured
low impact of the architecture to network functionality and verified its
backwards compatibility with existing applications. 

% Chapter~\ref{sec:homework} discussed issues related to management timescale traffic engineer-
% ing, i.e. dealing with aggregates of traffic between ISPs. It described current
% mechanisms within the Internet for performing this and discussed the relation
% with data timescale traffic engineering. It then looked in more detail at inter-
% AS traffic engineering using the BGP protocol, and proposed the price path at-
% tribute as a mechanism that improves the facility for management timescale
% traffic engineering. Design considerations for the price path attribute were
% then detailed and implementation within a BGP simulator described. Finally,
% results of simulations using this simulator were presented and discussed.
% Chapter~\ref{sec:signpost} presented the case for evolving from the state of the current In-
% ternet toward that presented in this dissertation. It began by describing the
% requirements users and operators have for Internet traffic engineering, and the
% state of the art of their implementation. Having previously demonstrated the
% benefits of the mechanisms presented in Chapters 3 and 4, arguments for and
% against the deployment process were presented. The deployment pro- cess itself
% was shown to be desirable; this was followed by a discussion of user and
% operator perceptions of service provision and a concrete example of the services
% that deployment of these mechanisms would allow. Finally, con- sequences from
% the point of view of the network and associated economic structures were
% presented.

\section{Future Work}
  
This section notes areas where further work is required, and future directions
related work could take. The first such area, and one which applies on the
results of Chapter~\ref{sec:sdn_scalability} is the definition of a distributed
scalable network control framework. Hierarchical control, as presented through
the evaluation experiment using the \sdnsim, has low impact on network
performance, but a complete architectural design requires further exploration of
mechanisms providing strong guarantees on control plane responsiveness. A series
of approaches have been proposed in the field, e.g. nicira edge-based control
architecture~\cite{koponen12}, provide a performable approach to reduce core
network control load, and the \sdnsim platforms provides an sufficient framework
to evaluate the impact of such control distribution mechanisms.

Additionally, a number of areas for further work arise from the work in
Chapter~\ref{sec:homework}. A first research effort is to evaluate the
scalability of the control scheme. The control logic is capable to handle
multiple switches, but there is still reasonable complexity to integrate other
network control protocol with the design of the router, in order to introduce
our control design in larger network settings. In addition, the novel control
architecture of our router provide novel opportunities to augment home network
functionality . Network design can be extended to enable seamless mobility to
home network users when connecting to remote wifi networks and the user can
carry his network configurations to the new network. On the other hand,
interesting opportunities arise if we extend the homework QoS model in the ISP
network. In parallel, a interested economical model can be developed to reuse
network resources  which remain unutilized by the homeowner for a period of
time, e.g. exchange high priority traffic for specific timeslots with
neighbours. 

Finally, a number of pieces of work arise from Chapter~\ref{sec:signpost}.  The
\signpost design at the moment is limited in  its policy expressiveness, which
though is sufficient to address our motivations.  Nonetheless, the
authentication primitives can provide a scalable and global mechanism to develop
novel network policy frameworks and address a series of problems stemming from
the inability of the network layer to authenticate network end-points.
Furthermore, the hierarchical structure of the \signpost architecture can be
used to reflect higher level social relationship and increase the in-network
flexibility. For example, the devices of a user who is a member of the University
of Cambridge can connect through the \signpost architecture with the university
network control plane,  when its devices connect to the university network. 
At run-time the device \signpost daemon can negotiate with the university
\signpost controller the device network requirements and reflect them in the
University network control. 
