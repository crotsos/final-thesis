\def\baselinestretch{1}
\chapter{Conclusions and Future Work} \label{sec:conclusions}

\def\baselinestretch{1.66}

This chapter concludes the dissertation by summarising the work it described,
and noting areas in which further work is required.

\section{Summary}

This dissertation has addressed issues of network performance scalability using
the SDN paradigm.  Chapter~\ref{s:introduction:introduction} began by
motivating the requirement to scale the functionality of existing network
protocols and technologies in order to support the design limitation which are
highlighted by the increase in network technology adoption.  We argued that
network performance is multi-dimensional (e.g.~resource control, management,
connectivity), and depends on the applications and the deployment environment.
We claimed that specialized control plane architectures can mitigate network
bottlenecks introduced by protocol design, while ensuring backwards
compatibility and high performance. Effective control plane evolution must
address the requirements and exploit the inherent opportunities of the deployed
environment. 

Chapter~\ref{ch:background} then considered background and related work to the
problem of network control. We provided a bottom-up design discussion on
available network control mechanisms. We elaborated on the architecture of
current network devices and presented a generic architecture model of the integration
between the control and data plane in a single device in order to highlight the
physical limitations in  network performance. Furthermore, we reviewed
the current production-level network control protocols and mechanisms for the
data link and network layers and it was argued that their ability for responsive,
flexible, and user friendly control is limited. Furthermore, we surveyed a
series of experimental approaches which address network control limitations and
provide flexible, distributed and evolvable control.  Namely we presented Active
Network, Devolved Control of ATM Networks and Software Defined Networking. In
order to highlight the opportunities provided by these mechanisms, we concluded
this chapter with an extensive presentation of novel control applications build
on top of the SDN paradigm. 

The bulk of the experimental contribution of this dissertation was reported in
the following three chapters. Chapter~\ref{sec:sdn_scalability} analysed the
elementary scalability of the SDN paradigm. In this chapter, we presented two
measurement platforms, enabling network experimenters to evaluate the
performance of SDN architectures. \oflops is a high-precision
hardware-accelerated \of switch measurement platform which enables
experimenters to understand the performance behaviour of \of-enabled devices,
and \sdnsim, a lightweight high-precision network simulation and emulation
framework. Using \oflops, we developed and ran a series of tests characterising
the elementary functionalities of the \of protocol and detected significant
variance between switch protocol implementations. \sdnsim is a high precision
experimentation environment, which allows users to implement their control
logic and traffic models and evaluate the performance of an SDN architecture.
The platform provides the ability to simulate, using the \ns{3} platform, or
emulate, using the Xen virtualisation platform, a experimental definition.  The
platform provides enhanced realism on the performance characteristics of the
network control plane, while through our evaluation we highlighted the achieved
experimental scalability. Using \sdnsim, we replicated the functionality of a
small-scale datacenter network and evaluated the effectiveness of control
centralisation.

Chapter~\ref{sec:homework} elaborated on the problem of management scalability in
modern home networks. Using ethnographic and measurement studies of the home
setting, we identified significant mismatches between the user-requirement and
user-understanding of network functionality, and the existing technologies.
Motivated by this observation, we redesigned the home router, implementing a
series of control modifications which enhanced user control and bridged the user
perception with the underlying network functionality. The proposed architecture
was extended with a collaborative resource control mechanism which integrated
user application-level requirements with the ISP policy, in an effort to
develop a user-friendly control scheme for the last-mile bottleneck in
residential broadband networks. We presented the development of a strawman
implementation of our system and verified that: the architectures incurs minimal
impact on network functionality; the protocol modifications remain
backwards compatible with a number of popular devices, OSes and applications; and
 the resource control mechanism improves the support for latency and
bandwidth sensitive applications in congested residential networks. 

Chapter~\ref{sec:signpost} discussed Internet-scale naming and connectivity
scalability.  Through the work of this chapter, we aimed to develop a
decentralised and Internet-wide federated network between the devices of a user.
We motivated our effort by presenting the limitations of existing approaches in
terms of usability and privacy, and argued that an evolved control plane for
end-hosts can address such limitations and support all required functionality.
We proposed the \signpost architecture providing secure, continuous and
decentralised inter-device connectivity. \signpost reuses existing connectivity
mechanisms to provide ad hoc end-to-end paths between devices. The system
provides a global naming structure and uses the DNS protocol to establish a
global control channel through which \signpost automates distributed evaluation
and configuration of connection-establishing mechanisms.  We presented a
strawman implementation  of \signpost and its integration with
\openvpn, TOR, SSH, Privoxy, NAT punching and DNS-SD mechanisms.  Using the
\signpost implementation, we characterised the low impact of the
architecture on network functionality and its backwards compatibility with
existing applications. 

\section{Contributions}

The dissertation makes the following three contributions:

\paragraph{Control plane scalability} 
This thesis presented the first scalability characterisation of \of
implementations. We highlighted the significant performance diversity between
implementations which can affect the performance and the correctness of control
architectures.  Furthermore, we developed \sdnsim, an experimentation platform
for SDN architectures, providing the ability to emulate and simulate network
experiments. \sdnsim provides high fidelity and scalability on replicating complex
network control architecture
and provides intuitive control between fidelity and scalability. 
Using \sdnsim we evaluate the performance of a hierarchical control architecture
in a small-scale datacenter.

\paragraph{Management scalability}

This thesis presented a novel control architecture establishing scalable
management for home network. We presented a flow-based controller, which exploited
the social conventions in the home to manage introduction of devices  to the
network, and their subsequent access to each other and Internet hosted services.
Additionally, we proposed a modification in the control architecture of the ISP
network, which enables users to express and enforce their resource requirements
in a user-friendly manner. We provided strong evidence on the scalability,
backwards compatibility and effectiveness of our solution.  

\paragraph{Connectivity and naming scalability}

This thesis analysed the significant limitation introduced by the current
Internet architecture on the connectivity ability between the devices of users.
In order to mitigate these limitations we presented \signpost, a decentralised
control architecture providing global names for user devices and continuous
connectivity between them. We presented the flexibility of \signpost in
encapsulating a wide range of connection-establishing mechanisms and provided
strong evidence on its backwards compatibility with existing applications, and its
performance scalability. 

\section{Future Work}

The experimental results and the practical solutions presented in the thesis
provide fruitful seeds to cultivate a wider research agenda on control plane
evolvability.

\subsection{Distributed Network Control}

One of the first use cases of the SDN paradigm was the centralisation of control in
order to improve policy effectiveness and ease network management.  Nonetheless this
vision has evolved and refocused on the development of distributed control
architectures. Control distribution is motivated by two observations.  Firstly,
load and latency of the control channel increases proportionally to the size of the
network, while reliability guarantees relax. Secondly,
defining one global control policy which is able to encapsulate multiple policy
aspects (e.g.~security, performance, access control) exhibits significant
complexity. Applications have shifted to a multi-controller paradigm using
either centralised proxies~\cite{flowvisor-osdi} or separating the network
in domains and using distribute algorithms to synchronise state between
controllers~\cite{Koponen10}. 

The work presented in Chapter~\ref{sec:sdn_scalability} provides a scalable
control experimentation platform, a powerful tool to understand further the
impact of  distributed design patterns on the performance of a network.
Hierarchical control, as presented through the evaluation experiment using the
\sdnsim, has low impact on network performance, but a complete architectural
design requires further evaluation of mechanisms with strong control plane
responsiveness and reliability guarantees. 

\subsection{User-centric Networking}

As we have discussed in Chapter~\ref{sec:homework}, current network technologies
exhibit a significant mismatch with the user requirements. The outcomes from the
previous two chapters have provided strong evidence on the ability of networks to
reconsider control and augment it with meaningful user input. Rather than
relegating users to an artefact of the application layer, accommodating users
and their relationships at all layers of the system can improve user
satisfaction and network functionality. We consider two main extensions on the
work of the thesis towards this goal. 

Firstly, a number of areas arise from the work in Chapter~\ref{sec:homework}.
The presented control architecture provides novel opportunities to augment home
network functionality and exploit the wider social context of the home setting.
In the local network, home guests can inject their configuration into the local
network policy and improve their experience. This approach is not limited in
the local network and can expand to wider contexts. For example, neighbours can
negotiate  resource control by taking advantage of the social context in a
neighbourhood. Users can coordinate socially to share un-utilized resources on
the last mile of the residential connection,  e.g.~exchange high priority
traffic for specific timeslots with neighbours.

Secondly, a number of pieces of work arise from Chapter~\ref{sec:signpost}.
The \signpost design at the moment is limited in its policy expressiveness, but
is sufficient to address our motivations.  Nonetheless, the authentication
primitives can provide a scalable and global mechanism to develop novel network
policy frameworks and address a series of problems stemming from the inability
of the network layer to authenticate network end-points. Furthermore, the
hierarchical structure of \signpost can be used to reflect higher level social
relationship and increase the in-network flexibility. Such a control
architecture can take advantage of recent theory frameworks, like Bigraphs, to
scale the respective complexity. 

% \section{Conclusion}
% 
% In summary, this dissertation has argued that network technologies exhibit
% significant limitation in scaling their functionality and support of the multiple
% aspects of network performance. We identified these limitations as due to the
% unforeseen functional requirements that occurred  by the wide adoption of network
% technologies. In order to address these limitations, we contend that network
% control must be redesigned and specialized in order to fit the requirements and
% take advantage of the properties of the deployment environment. This
% dissertation has presented and evaluated mechanisms to control plane
% functionality, management and connectivity scalability. 



